\subsection{Formation of $A$}
The formula for each element of $A$ is given by (\ref{eq:a_definition}). This formulation requires the calculation
of both $\nabla \phi_i$ and $\nabla \phi_j$. However, since every $\phi$ is piece-wise linear, the gradient
is constant over a single triangular region (and in most places zero). Within a particular triangle $T_{kji}$, applying the results from
(\ref{eq:plane_solution}) gives
\begin{equation} \label{eq:grad_phi_definition}
\nabla \phi_k = \mat{a_{kji} \\ b_{kji}}
\end{equation}

For any $\phi_k$ and $\phi_j$, $j \ne k$, they will share either zero, one, or two triangular regions. If they share
zero, the gradient of either $\phi_k$ or $\phi_j$ will be zero over the domain, forcing (\ref{eq:a_definition}) to
be zero. If they share only a single region, it is because they lie on the domain boundary, which would imply that
they are zero by the boundary conditions. The last case, sharing two triangular regions implies that they are interior
nodes that are adjacent on the mesh. Because they are adjacent, in the two shared triangular regions both of their
gradients will be non-zero. Furthermore, if $\nabla \phi_k = \mat{a_k \\ b_k}$ and $\nabla \phi_j = \mat{a_j \\ b_j}$
inside of some triangle $T$, we have
\begin{equation}
\int\limits_T \nabla \phi_j \cdot \nabla \phi_k dT
=
\left( a_k a_j + b_k b_j \right) \mathop{Area}(T)
\end{equation}
Summing over both triangles gives a very straightforward method for calculating each individual element of A. Furthermore,
A will be symmetric, since
\begin{equation}
\innerproductdot{\phi_k}{\phi_j} = \innerproductdot{\phi_j}{\phi_k}
\end{equation}
so further computation can be saved.
