\subsection{Defining $\phi_k$}
Since $\phi_k \in H^1$, we know that it is piecewise linear. We will choose $\phi_k$ to be zero at all nodes
$N_1, N_2, \ldots$ but be 1 at $N_k$. This will cause $\phi_k$ to be tent-shaped. Because $\phi_k$ is piece-wise linear,
in order to find a closed form we need only consider each "piece", which is this case a triangular region.

Consider a single triangle with nodes at $N_k$, $N_j$, and $N_i$. Given our definition of $\phi_k$, we know $\phi_k(N_k) = 1$,
but 0 for any other node. So in the region defined by our triangle (we'll call it $T_{kji}$), $\phi_k$ is non-zero. Because
$\phi_k$ is piece-wise linear, that means $\phi_k$ is planar on the region, and that the plane passes through these three points:
$(x_k, y_k, 1)$, $(x_j, y_j, 0)$, $(x_i, y_i, 0)$. Considering a plane in three dimensions can be written as
\begin{equation} \label{eq:plane}
z = P(x, y) = ax + bx + c
\end{equation}
we can form a linear system to solve for our cooefficients:
\begin{equation} \label{eq:plane_solution}
\mat{
    x_k & y_k & 1 \\
    x_j & y_j & 1 \\
    x_i & y_i & 1
}
\mat{a \\ b \\ c}
=
\mat{1 \\ 0 \\ 0}
\end{equation}
Given the size of the matrix a closed form solution can be used (CITE), or direct solver can be used.
