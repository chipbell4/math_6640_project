\subsection{Derivation of the Finite Element Solution}
We first consider the weak form of (\ref{eq:main_pde}) by approximating $u$ as $u_h \in H^1(\Omega)$. Assuming we have
a triangulation of $\Omega$, consisting of $N$ triangles, we can choose 
$\phi_1(x,y), \phi_2(x,y), \ldots, \phi_N(x,y)$ as basis functions for $H^1(\Omega)$, we can write $u_h$ as linear
combination of those basis elements:
\begin{equation} \label{eq:u_h}
u_h(x,y,t) = \sum\limits_{k=1}^N c_k(t) \phi_k(x,y)
\end{equation}
Note that at a particular time $t$, $c_k(t)$ will give the height of a wave at node $k$. So, by solving each of the values
of $c_k(t)$ over time, we will obtain the values of $u_h$ at time $t$.

The weak form of $u_h$ is given by taking the inner product 
\begin{equation} \label{eq:u_h_weak_form}
\innerproduct{ \left( \secondpartial{u_h}{t} - \gamma \firstpartial{u_h}{t} \right) } { v }
=
\beta^2 \innerproductdot{u}{v(x,y)}
+ \innerproduct{F}{v}
\end{equation}
for any $v \in H^1(\Omega)$. Since this works for any $v$, we can choose $v$ to be any of the basis functions $\phi$
we chose for $H^1(\Omega)$. So, for any basis $\phi_j$, we can expand $u_h$ using (\ref{eq:u_h}) to obtain a new form:
\begin{equation} \label{eq:u_h_weak_form_expanded}
\sum\limits_{k=1}^N \left( \secondderivative{c_k}{t} - \gamma\firstderivative{c_k}{t} \right) \innerproduct{\phi_k}{\phi_j}
=
\beta^2\sum\limits_{k=1}^N c_k(t) \innerproductdot{\phi_k}{\phi_j}
+
\innerproduct{F}{\phi_j}
\end{equation}

We now let $M$ be the matrix of pairwise inner products of the $\phi_k$'s, for example
\begin{equation} \label{eq:m_definition}
m_{ij} = \innerproduct{\phi_i}{\phi_j}
\end{equation}

And we let $A$ be the matrix of pairwise inner products of $\nabla \phi_k$, for example
\begin{equation} \label{eq:a_definition}
a_{ij} = \innerproductdot{\phi_i}{\phi_j}
\end{equation}

Also, let $\vec{c}(t)$ be the vector with each entry $c_k(t)$ for $k=1, 2, \ldots$. Lastly, define
$\vec{F}(t)$ to be a vector, s.t. the $k$th entry is given as 
\begin{equation} \label{eq:vec_F_definition}
\innerproduct{F}{\phi_k}
\end{equation}

We can now rewrite (\ref{eq:u_h_weak_form_expanded}) as
\begin{equation} \label{eq:u_h_matrix}
M\left( \secondderivative{\vec{c}}{t} - \gamma\firstderivative{\vec{c}}{t} \right) = \beta^2 A \vec{c} + \vec{F}
\end{equation}

We now can build a finite difference scheme for $\vec{c}$ by using the following expansion for the second derivative:
\begin{equation} \label{eq:second_derivative_finite_difference}
\secondderivative{c}{t} \approx \secondfinitediff{c}
\end{equation}
and the following (second order) expansion for the first derivative
\begin{equation} \label{eq:first_derivative_finite_difference}
\firstderivative{c}{t} \approx \firstfinitediff{c}
\end{equation}

Substituting into (\ref{eq:u_h_matrix}) gives us
\begin{equation} \label{eq:finite_diff_weak_form}
M \left( \secondfinitediff{c} - \gamma \firstfinitediff{c} \right) = \beta^2 A c^n + \vec{F}
\end{equation}

Lastly, we can rearrange and solve for $c^{n+1}$
\begin{equation} \label{eq:finite_diff_solution}
c^{n+1} = \step{-}^{-1} M^{-1}
\left(
\left( \frac{2}{\delta t^2}M + \beta^2 A \right) c^n
+
\step{+} M c^{n-1}
+
\vec{F}
\right)
\end{equation}

Given this formulation, the only remaining challege is the numerical calculation of $M$, $A$ and potentially
$M^{-1}$ (unless it is implicitly calculated every step via a banded matrix solver).
\cite{greenwade93}
