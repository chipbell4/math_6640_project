\subsection{Formation of $M$}
To calculate the elements of $M$, the inner product of corresponding basis functions of $\Omega$ must be calculated
(see (\ref{eq:m_definition})). Since the basis functions are 0 for most of the domain, and most nodes are not
connected, we expect $M$ to be very sparse. So given a triangular mesh, we need only consider two cases: The inner
product of two adjacent interior node basis function (since exterior nodes are 0), or the inner product of an interior
node basis function with itself.

As mentioned before, two adjacent interior node basis functions $\phi_k$ and $\phi_h$ will always share two triangular
regions of $\Omega$ in which they are both non-zero. Considering just one of these triangles $T$, we can find a solution method for a
single region and simply extrapolate up to using two regions. For just the single triangle $T$ defined by the points
$N_k$, $N_j$, and $N_i$, we are wanting to calculate
\begin{equation}
\iint\limits_T \phi_k \phi_j dT
\end{equation}
The problem with attempting to calculate this integral directly is that $T$ may be an very domain to integrate over
directly. Instead, we can apply an affine transformation to $T$, so to transform the region into a $(u,v)$ space that
has more straight-forward limits of integration. We'll refer to the transformation as $S$.

We'll choose $S$ to be a transformation that takes a triangle with corners at $(0,0)$, $(1,0)$, and $(1,1)$. and
transforms them to $T$. We can choose any ordering, but we'll choose $S^{-1}N_k = (0, 0)$, $S^{-1}N_j = (1, 0)$,
and $S^{-1}N_i = (1, 1)$. Note that
by transforming to this new $(u,v)$ space our integral becomes
\begin{equation}
\int\limits_0^1 \int\limits_0^u \phi_k(S(u, v)) \phi_j(S(u, v)) |S| dv du
\end{equation}
using a straight-forward change of variables (CITE).

Since $S$ is affine, we'll need to use homogeneous coordinates to treat as a linear transformation (CITE). Given a homogeneous
formulation, $S$ takes the form
\begin{equation}
\mat{ a & b & c \\ d & e & f \\ 0 & 0 & 1 }
\end{equation}

Given the mapping we chose earlier, we can form an equation
\begin{equation}
S \mat{ 0 & 1 & 1 \\ 0 & 0 & 1 \\ 1 & 1 & 0 }
= \mat{ x_k & x_j & x_i \\ y_k & y_j & y_i \\ 1 & 1 & 1}
\end{equation}
and then rearrange to create a linear system for which we can solve the coefficients of $S$:
\begin{equation}
\mat{
    0 & 0 & 1 & 0 & 0 & 0 \\
    1 & 0 & 1 & 0 & 0 & 0 \\
    1 & 1 & 0 & 0 & 1 & 0 \\
    0 & 0 & 0 & 0 & 0 & 1 \\
    0 & 0 & 0 & 1 & 0 & 1 \\
    0 & 0 & 0 & 1 & 1 & 0 
}
\mat{a \\ b \\ c \\ d \\ e \\ f}
=
\mat{x_k \\ x_j \\ x_i \\ y_k \\ y_j \\ y_i }
\end{equation}
Given the form of the matrix, this is easily solved by hand, and we have
\begin{equation}
S = \mat{
    x_j - x_k & x_k - x_j + x_i & x_k \\
    y_j - y_k & y_k - y_j + y_i & y_k \\
    0 & 0 & 1
}
\end{equation}
Moreover, $S$ is a simple enough form to find a closed solution for $|S|$:
\begin{equation}
|S| = 
(x_j - x_k)(y_k - y_j + y_i) -
(y_j - y_k)(x_k - x_j + x_i)
\end{equation}

Lastly, we need to find a closed form of $\phi_k(S(u,v))\phi_j(S(u,v))$, and ultimately a closed form for the full
integral. Using $S$, $\phi_k$ will now take the form
\begin{equation}
\phi_k(S(u,v)) = A_k ((x_j - x_k)u + (x_k - x_j + x_i)v + x_k) + B_k((y_j - y_k)u + (y_k - y_j + y_i)v + y_k)
\end{equation}
After some algebra we get
\begin{equation}
((A_k(x_j-x_k) + B_k(y_j-y_k))u + (A_k(x_k-x_j+x_i) + B_k(y_k-y_j+y_i))v + (A_k x_k + B_k y_k)
\end{equation}
Or for simplicity we'll right it as
\begin{equation}
A'_k u + B'_k v + C'
\end{equation}
Multiplying $\phi_k$ and $\phi_j$ gives us
\begin{equation}
(A'_k A'_j)u^2 + (B_k'B_j')v^2 + (A'_k B'_j + B'_k A'_j)uv + (A'_k C'_j + C'_k A'_j)u
+ (B'_kC'_j + C'_kB'_j)v + (C'_k C'_j)
\end{equation}
Now, we can integrate
\begin{equation}
\int\limits_0^1\int\limits_0^u \phi_k(S(u,v)) \phi_j(S(u,v)) dvdu
\end{equation}
Which gives us a final closed form
\begin{equation}
\frac{A'_k A'_j}{4} + \frac{B_k'B_j'}{12} + \frac{A'_k B'_j + B'_k A'_j}{12} +
\frac{A'_k C'_j + C'_k A'_j + B'_kC'_j + C'_kB'_j}{6} + \frac{C'_k C'_j}{2}
\end{equation}
for the integral over a single triangle. We can apply this method twice for the two shared triangles to calculate
$\innerproduct{\phi_k}{\phi_j}$.

For $\innerproduct{\phi_k}{\phi_k}$, the method is exactly the same, but we must integrate over all triangles with a
vertex at $N_k$, since $\phi_k$ shares all neighboring triangles with itself.
