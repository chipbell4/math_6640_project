\subsection{Formation of $M$}
To calculate the elements of $M$, the inner product of corresponding basis functions of $\Omega$ must be calculated
(see (\ref{eq:m_definition})). Since the basis functions are 0 for most of the domain, and most nodes are not
connected, we expect $M$ to be very sparse. So given a triangular mesh, we need only consider two cases: The inner
product of two adjacent interior node basis function (since exterior nodes are 0), or the inner product of an interior
node basis function with itself.

As mentioned before, two adjacent interior node basis functions $\phi_k$ and $\phi_h$ will always share two triangular
regions of $\Omega$ in which they are both non-zero. Considering just one of these triangles $T$, we can find a solution method for a
single region and simply extrapolate up to using two regions. For just the single triangle $T$ defined by the points
$N_k$, $N_j$, and $N_i$, we are wanting to calculate
\begin{equation}
\iint\limits_T \phi_k \phi_j dT
\end{equation}
The problem with attempting to calculate this integral directly is that $T$ may be an very domain to integrate over
directly. Instead, we can apply an affine transformation to $T$, so to transform the region into a $(u,v)$ space that
has more straight-forward limits of integration. We'll refer to the transformation as $S$.

We'll choose $S$ to be a transformation that takes $T$ and transforms it to a triangle with corners at $(0,0)$, $(1,0)$,
and $(1,1)$. We can choose any ordering, but we'll choose $SN_k = (0, 0)$, $SN_j = (1, 0)$, and $SN_i = (1, 1)$. Note that
by transforming to this new $(u,v)$ space our integral becomes
\begin{equation}
\frac{1}{|S|} \int\limits_0^1 \int\limits_0^u S \phi_k \phi_j dv du
\end{equation}
using a straight-forward change of variables (CITE).

Since $S$ is affine, we'll need to use homogeneous coordinates to treat as a linear transformation (CITE). Given a homogeneous
formulation, $S$ takes the form
\begin{equation}
\mat{ a & b & c \\ d & e & f \\ 0 & 0 & 1 }
\end{equation}

Given the mapping we chose earlier, we can form an equation
\begin{equation}
S \mat{ x_k & x_j & x_i \\ y_k & y_j & y_i \\ 1 & 1 & 1}
=
\mat{ 0 & 0 & 1 \\ 0 & 1 & 1 \\ 1 & 1 & 1 }
\end{equation}
and then rearrange to create a linear system for which we can solve the coefficients of $S$:
\begin{equation}
\mat{
    x_k & y_k & 1 & 0 & 0 & 0 \\
    x_j & y_j & 1 & 0 & 0 & 0 \\
    x_i & y_i & 1 & 0 & 0 & 0 \\
    0 & 0 & 0 & x_k & y_k & 1 \\
    0 & 0 & 0 & x_j & y_j & 1 \\
    0 & 0 & 0 & x_i & y_i & 1
}
\mat{a \\ b \\ c \\ d \\ e \\ f}
=
\mat{0 \\ 0 \\ 1 \\ 0 \\ 1 \\ 1}
\end{equation}
Since this matrix is in block form, we can simply use a closed form for the $3 \times 3$ case (CITE) or
actually apply a direct solver.

Given that we now have $S$, and we know the form of $\phi_k$ and $\phi_j$ we now need to determine the form of
$S\phi_k \phi_j$ so that we can find a closed form for our inner product.

- Mention only two triangles shared really matters
- Integrating $\phi_k$ and $\phi_j$ over region sucks
- Perform a change of variables transformation and then integrate over a nicer region (cite this method)
    - Derive transformation matrix given $N_i, N_j, N_k$
    - Multiply $\phi_j \phi_k$ by the transformation, converting to $(u,v)$ coords. Don't forget the Jacobian.
    - Perform integration, find generic form for single triangle
    - Generalize to be able to handle two triangles (shared) between nodes.
