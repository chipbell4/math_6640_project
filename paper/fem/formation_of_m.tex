\subsection{Formation of $M$}
To calculate the elements of $M$, the inner product of corresponding basis functions of $\Omega$ must be calculated
(see (\ref{eq:m_definition})). Since the basis functions are 0 for most of the domain, and most nodes are not
connected, we expect $M$ to be sparse in general. So given a triangular mesh, we need only consider two cases: The inner
product of two adjacent interior node basis function (since exterior nodes are 0), or the inner product of an interior
node basis function with itself.

Two adjacent interior node basis functions $\phi_k$ and $\phi_j$ will only share two triangular
regions of $\Omega$ in which they are both non-zero since they can only share a single edge. Considering just one of these
triangles $T$, we can find a solution method for a
single region and simply replicate the process for the second region. For just the single triangle $T$ defined by the points
$N_k$, $N_j$, and $N_i$, we are wanting to calculate
\begin{equation}
\iint\limits_T \phi_k \phi_j dT
\end{equation}
The problem with attempting to calculate this integral directly is that $T$ may be an very difficult domain to integrate over
directly. Instead, we can apply a transformation to $T$, so to transform the region into a $(u,v)$ space that
has more straight-forward limits of integration. We'll refer to the transformation as $S$.

We'll choose $S$ to be a transformation that takes a triangle with corners at $(0,0)$, $(1,0)$, and $(0,1)$. and
transforms them to $T$, translated to the origin. We can choose any ordering, but for convenience we'll choose 
$S(0, 0) = (0,0)$, $S(1, 0) = N_k - N_i$, and $S(0, 1) = N_j - N_i$. This transformation shifts chooses $T$ to be
translated to the origin around $N_i$. Note that
by transforming to this new $(u,v)$ space our basis functions become considerably simpler:
\begin{equation}
S^{-1}\phi_k = u
\end{equation}
\begin{equation}
S^{-1}\phi_j = v
\end{equation}
As a result, our inner product simplifies too
\begin{equation}
\int\limits_0^1 \int\limits_0^{1-u} uv |S| dv du = \frac{|S|}{24}
\end{equation}
using a straight-forward change of variables \cite{change_of_variables}.

Calculating $S$ is very straightforward now, given our choice of points
\begin{equation}
S = \mat{
    x_{N_k} - x_{N_i} & x_{N_j} - x_{N_i} \\ 
    y_{N_k} - y_{N_i} & y_{N_j} - y_{N_i}
}
\end{equation}

Finally, $|S|$ is easily calculated, and as a note is equal to the twice the area of
the triangle $T$.

In order to fully calculate the inner product of all of $\phi_k$ and $\phi_j$, we simply need to repeat the process
for both triangles shared by $\phi_k$ and $\phi_j$.

For $\innerproduct{\phi_k}{\phi_k}$, the integral takes a different form
\begin{equation}
\innerproduct{\phi_k}{\phi_k} = \int\limits_0^1 \int\limits_0^{1-u} u^2 |S| dv du
\end{equation}
This is still an easy integral and gives a final result of $\frac{|S|}{12}$ for the shared node. We simply integrate
over all triangles connected to the node to calculate the full inner product.
