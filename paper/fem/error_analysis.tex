\subsection{Error Analysis}
It is certainly useful to know the stability of the scheme chosen in \label{eq:finite_diff_solution}, so that we can
choose appropriate time step values to prevent numerical instability. We'll first consider apply some algebra to
\label{eq:finite_diff_solution}, which gives us a somewhat simpler form
\begin{equation}
c^{n+1} = 
\frac{1}{2 + \gamma\delta t}
\left(
    \left( 4I - 2\alpha\delta t^2I - 2\beta\delta t^2M^{-1}A \right) c^n
    +
    \left( 2 - \gamma\delta t \right) c^{n-1}
\right)
\end{equation}

For the purposes of our error analysis, we'll assume that no force is being applied, since in our simulation it's
assumed the user won't provide force the entire time. Furthermore, we couldn't enforce stability if the system is
constantly having energy added to it. So, considering the error at time step $n+1$ and applying the triangle
inequality, we have
\begin{equation}
|\epsilon^{n+1}|
\le
\frac{1}{2 + \gamma\delta t}
\left(
    |4I - 2\alpha\delta t^2I - 2\beta^2\delta t^2M^{-1}A| |\epsilon^n|
    +
    |2 - \gamma\delta t| |\epsilon^{n-1}| 
\right)
\end{equation}
In the worst case, the error is not decreasing, which gives us $|\epsilon^{n-1}| = |\epsilon^n|$, which allows us to
combine terms, which gives us the following
\begin{equation}
\frac{|\epsilon^{n+1}|}{|\epsilon^n|}
=
\frac{1}{2 + \gamma\delta t} \left(
|4I - 2\alpha\delta t^2I - 2\beta^2\delta t^2M^{-1}A| + |2 - \gamma\delta t|
\right)
\end{equation}

This value must be less than or equal to $1$ in order for the scheme to be stable. Using the reverse triangle
inequality \cite{reverse_triangle_inequality}, we can expand each term to give us
\begin{equation}
\frac{6 - 2\alpha\delta t^2 - 2\beta^2 \delta t^2 |M^{-1}| |A| - \gamma\delta t}{2 + \gamma\delta t} \le 1
\end{equation}

We can omit some absolute values provided $\alpha$ and $\beta^2$ are not on the order of $\frac{1}{\delta t^2}$ and
$\gamma$ is not on the order of $\frac{1}{\delta t}$. We will assume this is true. We can rearrange to give a
quadratic inequality
\begin{equation}
0 \le \left(\alpha + \beta^2|M^{-1}| |A|\right)\delta t^2 + \gamma\delta t - 2
\end{equation}

Applying the quadratic equation gives us a lower bound for $\delta t$:
\begin{equation}
\delta t \ge \frac{-\gamma + \sqrt{\gamma^2 + 8(\alpha + \beta^2|M^{-1}| |A|)}}{2\alpha + 2\beta^2|M^{-1}| |A|}
\end{equation}

This result is surprising, but understandable. All terms in the scheme that contain $\delta t$ are negative, so we must
enforce that $\delta t$ is large enough to keep the total scale from growing too quickly.
