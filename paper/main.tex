\documentclass[a4paper,12pt]{article}

\usepackage{amsmath, amsthm, amssymb}

\begin{document}

% some custom commands
\newcommand{\innerproduct}[2]{\int\limits_{\Omega} #1 #2 d\Omega}
\newcommand{\innerproductdot}[2]{\int\limits_{\Omega} \nabla #1 \cdot \nabla #2 d\Omega}
\newcommand{\firstderivative}[2]{\frac{d #1}{d #2}}
\newcommand{\secondderivative}[2]{\frac{d^2 #1}{d #2^2}}
\newcommand{\firstpartial}[2]{\frac{\partial #1}{\partial #2}}
\newcommand{\secondpartial}[2]{\frac{\partial^2 #1}{\partial #2^2}}
\newcommand{\laplacian}[1]{\Delta #1}
\newcommand{\secondfinitediff}[1]{\frac{#1^{n+1} - 2#1^n + #1^{n-1}}{\delta t^2}}
\newcommand{\firstfinitediff}[1]{\frac{#1^{n+1} - #1^{n-1}}{2\delta t}}
\newcommand{\step}[1]{\left( \frac{1}{\delta t^2} #1 \frac{\gamma}{2\delta t} \right)}
\newcommand{\mat}[2][rrrr]{
    \left(\begin{array}{#1}
    #2 \\
    \end{array}
    \right)
}

\section{The main PDE}
The partial differential equation models a damped membrane over a region $\Omega$ bounded by a closed curve $\Gamma$.
\begin{equation} \label{eq:main_pde}
\secondpartial{u}{t} + \gamma \firstpartial{u}{t}
=
\beta^2 \laplacian{u} + F(x,y,t)
\end{equation}

where $u$ is the wave height at some point $(x,y)$ at some time $t$, $\gamma$ is a damping factor for the wave,
$\beta$ is the wave speed, and $F$ is a forcing function on the wave. Because this a drum head, (or ``membrane''), we choose
$u$ along $\Gamma$ to be 0, giving us a Dirichet boundary condition.

\section{Finite Element Analysis}
\subsection{Derivation of the Finite Element Solution}
We first consider the weak form of (\ref{eq:main_pde}) by approximating $u$ as $u_h \in H^1(\Omega)$. Assuming we have
a triangulation of $\Omega$, consisting of $N$ triangles, we can choose 
$\phi_1(x,y), \phi_2(x,y), \ldots, \phi_N(x,y)$ as basis functions for $H^1(\Omega)$, we can write $u_h$ as linear
combination of those basis elements:
\begin{equation} \label{eq:u_h}
u_h(x,y,t) = \sum\limits_{k=1}^N c_k(t) \phi_k(x,y)
\end{equation}
Note that at a particular time $t$, $c_k(t)$ will give the height of a wave at node $k$. So, by solving each of the values
of $c_k(t)$ over time, we will obtain the values of $u_h$ at time $t$.

The weak form of $u_h$ is given by taking the inner product 
\begin{equation} \label{eq:u_h_weak_form}
\innerproduct{ \left( \secondpartial{u_h}{t} + \gamma \firstpartial{u_h}{t} \right) } { v }
=
\beta^2 \innerproductdot{u}{v(x,y)}
+ \innerproduct{F}{v}
\end{equation}
for any $v \in H^1(\Omega)$. Since this works for any $v$, we can choose $v$ to be any of the basis functions $\phi$
we chose for $H^1(\Omega)$. So, for any basis $\phi_j$, we can expand $u_h$ using (\ref{eq:u_h}) to obtain a new form:
\begin{equation} \label{eq:u_h_weak_form_expanded}
\sum\limits_{k=1}^N \left( \secondderivative{c_k}{t} + \gamma\firstderivative{c_k}{t} \right) \innerproduct{\phi_k}{\phi_j}
=
\beta^2\sum\limits_{k=1}^N c_k(t) \innerproductdot{\phi_k}{\phi_j}
+
\innerproduct{F}{\phi_j}
\end{equation}

We now let $M$ be the matrix of pairwise inner products of the $\phi_k$'s, for example
\begin{equation} \label{eq:m_definition}
m_{ij} = \innerproduct{\phi_i}{\phi_j}
\end{equation}

And we let $A$ be the matrix of pairwise inner products of $\nabla \phi_k$, for example
\begin{equation} \label{eq:a_definition}
a_{ij} = \innerproductdot{\phi_i}{\phi_j}
\end{equation}

Also, let $\vec{c}(t)$ be the vector with each entry $c_k(t)$ for $k=1, 2, \ldots$. Lastly, define
$\vec{F}(t)$ to be a vector, s.t. the $k$th entry is given as 
\begin{equation} \label{eq:vec_F_definition}
\innerproduct{F}{\phi_k}
\end{equation}

We can now rewrite (\ref{eq:u_h_weak_form_expanded}) as
\begin{equation} \label{eq:u_h_matrix}
M\left( \secondderivative{\vec{c}}{t} + \gamma\firstderivative{\vec{c}}{t} \right) = \beta^2 A \vec{c} + \vec{F}
\end{equation}

We now can build a finite difference scheme for $\vec{c}$ by using the following expansion for the second derivative:
\begin{equation} \label{eq:second_derivative_finite_difference}
\secondderivative{c}{t} \approx \secondfinitediff{c}
\end{equation}
and the following (second order) expansion for the first derivative
\begin{equation} \label{eq:first_derivative_finite_difference}
\firstderivative{c}{t} \approx \firstfinitediff{c}
\end{equation}

Substituting into (\ref{eq:u_h_matrix}) gives us
\begin{equation} \label{eq:finite_diff_weak_form}
M \left( \secondfinitediff{c} + \gamma \firstfinitediff{c} \right) = \beta^2 A c^n + \vec{F}
\end{equation}

Lastly, we can rearrange and solve for $c^{n+1}$
\begin{equation} \label{eq:finite_diff_solution}
c^{n+1} = \step{+}^{-1} M^{-1}
\left(
\left( \frac{2}{\delta t^2}M + \beta^2 A \right) c^n
+
\step{-} M c^{n-1}
+
\vec{F}
\right)
\end{equation}

Given this formulation, the only remaining challege is the numerical calculation of $M$, $A$ and potentially
$M^{-1}$ (unless it is implicitly calculated every step via a banded matrix solver).
\cite{greenwade93}

\subsection{Defining $\phi_k$}
Since $\phi_k \in H^1$, we know that it is piecewise linear. We will choose $\phi_k$ to be zero at all nodes
$N_1, N_2, \ldots$ but be 1 at $N_k$. This will cause $\phi_k$ to be tent-shaped. Because $\phi_k$ is piece-wise linear,
in order to find a closed form we need only consider each "piece", which is this case a triangular region.

Consider a single triangle with nodes at $N_k$, $N_j$, and $N_i$. Given our definition of $\phi_k$, we know $\phi_k(N_k) = 1$,
but 0 for any other node. So in the region defined by our triangle (we'll call it $T_{kji}$), $\phi_k$ is non-zero. Because
$\phi_k$ is piece-wise linear, that means $\phi_k$ is planar on the region, and that the plane passes through these three points:
$(x_k, y_k, 1)$, $(x_j, y_j, 0)$, $(x_i, y_i, 0)$. Considering a plane in three dimensions can be written as
\begin{equation} \label{eq:plane}
z = P(x, y) = ax + bx + c
\end{equation}
we can form a linear system to solve for our cooefficients:
\begin{equation} \label{eq:plane_solution}
\mat{
    x_k & y_k & 1 \\
    x_j & y_j & 1 \\
    x_i & y_i & 1
}
\mat{a \\ b \\ c}
=
\mat{1 \\ 0 \\ 0}
\end{equation}
Given the size of the matrix a closed form solution can be used (CITE), or direct solver can be used.

\subsection{Formation of $A$}
The formula for each element of $A$ is given by (\ref{eq:a_definition}). This formulation requires the calculation
of both $\nabla \phi_i$ and $\nabla \phi_j$. However, since every $\phi$ is piece-wise linear, the gradient
is constant (and in most places zero). Within a particular triangle $T_{kji}$, applying the results from
(\ref{eq:plane_solution}) gives
\begin{equation} \label{eq:grad_phi_definition}
\nabla \phi_k = \mat{a_{kji} \\ b_{kji}}
\end{equation}

For any $\phi_k$ and $\phi_j$, $j \ne k$, they will share either zero, one, or two triangular regions. If they share
zero, the gradient of either $\phi_k$ or $\phi_j$ will be zero over the domain, forcing (\ref{eq:a_definition}) to
be zero. If they share only a single region, it is because they lie on the domain boundary, which would imply that
they are zero by the boundary conditions. The last case, sharing two triangular regions implies that they are interior
nodes that are adjacent on the mesh. Because they are adjacent, in the two shared triangular regions both of their
gradients will be non-zero. Furthermore, if $\nabla \phi_k = \mat{a_k \\ b_k}$ and $\nabla \phi_j = \mat{a_j \\ b_j}$
inside of some triangle $T$, we have
\begin{equation}
\int\limits_T \nabla \phi_k \nabla \phi_j
=
\left( a_k a_j + b_k b_j \right) \mathop{Area}(T)
\end{equation}
Summing over both triangles gives a very straightforward method for calculating each individual element of A. Furthermore,
A will be symmetric, since
\begin{equation}
\innerproductdot{\phi_k}{\phi_j} = \innerproductdot{\phi_j}{\phi_k}
\end{equation}
so further computation can be saved.

\subsection{Formation of $M$}
\subsection{Formation of $M^{-1}$}

\bibliographystyle{plain}
\bibliography{main}

\end{document}
