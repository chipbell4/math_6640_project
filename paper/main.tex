\documentclass[a4paper,12pt]{article}

\usepackage{amsmath, amsthm, amssymb, url, hyperref}

\begin{document}

% some custom commands
\newcommand{\innerproduct}[2]{\int\limits_{\Omega} #1 #2 d\Omega}
\newcommand{\innerproductdot}[2]{\int\limits_{\Omega} \nabla #1 \cdot \nabla #2 d\Omega}
\newcommand{\firstderivative}[2]{\frac{d #1}{d #2}}
\newcommand{\secondderivative}[2]{\frac{d^2 #1}{d #2^2}}
\newcommand{\firstpartial}[2]{\frac{\partial #1}{\partial #2}}
\newcommand{\secondpartial}[2]{\frac{\partial^2 #1}{\partial #2^2}}
\newcommand{\laplacian}[1]{\Delta #1}
\newcommand{\secondfinitediff}[1]{\frac{#1^{n+1} - 2#1^n + #1^{n-1}}{\delta t^2}}
\newcommand{\firstfinitediff}[1]{\frac{#1^{n+1} - #1^{n-1}}{2\delta t}}
\newcommand{\step}[1]{\left( \frac{1}{\delta t^2} #1 \frac{\gamma}{2\delta t} \right)}
\newcommand{\mat}[2][rrrrrr]{
    \left(\begin{array}{#1}
    #2 \\
    \end{array}
    \right)
}

\title{A Real Time Finite Element Simulation of a Damped/Forced Surface Wave on a Membrane using WebGL}
\author{Chip Bell}
\date{December 2014}
\maketitle

\begin{abstract}
In this project, a finite element model is developed for a modified wave equation that includes both a damping term
and a forcing term. A demonstration was built using various modern web technologies, including JavaScript and WebGL.
Later sections describe those technologies, explain how they were used in building the simulation, and mention some
of the issues encountered, along with how they were overcome or avoided altogether.
\end{abstract}


\section{The main PDE}
For my project I'll be modeling a modified form of the wave equation that encorporates damping, elasticity, and
forcing. I'll solve the equation over a domain $\Omega$ in 2 dimensions with a Dirichlet boundary $\Gamma$. 
\begin{equation} \label{eq:main_pde}
\secondpartial{u}{t} + \gamma \firstpartial{u}{t} + \alpha u
=
\beta^2 \laplacian{u} + F(x,y,t)
\end{equation}
where $u$ is the wave height at some point $(x,y)$ at some time $t$, $\gamma \ge 0$ is a damping factor for the wave,
$\beta$ is the wave speed, $\alpha$ is an elasticity term and $F$ is a forcing function on the wave. This equation
will model a taunt drum head over the domain $\Omega$, our intended purpose but is also known as the
\emph{Telegraph Equation} and has analytic solutions \cite{pde_solution}. Most derivations I have found approach
the problem from a electric circuit standpoint. One such derivation for the one dimensional case can be found in
\cite{pde_derivation}.

\section{Finite Element Analysis}
\subsection{Derivation of the Finite Element Solution}
We first consider the weak form of (\ref{eq:main_pde}) by approximating $u$ as $u_h \in H^1(\Omega)$. Assuming we have
a triangulation of $\Omega$, consisting of $N$ triangles, we can choose 
$\phi_1(x,y), \phi_2(x,y), \ldots, \phi_N(x,y)$ as basis functions for $H^1(\Omega)$, we can write $u_h$ as linear
combination of those basis elements:
\begin{equation} \label{eq:u_h}
u_h(x,y,t) = \sum\limits_{k=1}^N c_k(t) \phi_k(x,y)
\end{equation}
Note that at a particular time $t$, $c_k(t)$ will give the height of a wave at node $k$. So, by solving each of the values
of $c_k(t)$ over time, we will obtain the values of $u_h$ at time $t$.

The weak form of $u_h$ is given by taking the inner product 
\begin{equation} \label{eq:u_h_weak_form}
\innerproduct{ \left( \secondpartial{u_h}{t} + \gamma \firstpartial{u_h}{t} + \alpha u \right) } { v }
=
- \beta^2 \innerproductdot{u}{v}
+ \innerproduct{F}{v}
\end{equation}
for any $v \in H^1(\Omega)$. Since this works for any $v$, we can choose $v$ to be any of the basis functions $\phi$
we chose for $H^1(\Omega)$. So, for any basis $\phi_j$, we can expand $u_h$ using (\ref{eq:u_h}) to obtain a new form:
\begin{equation} \label{eq:u_h_weak_form_expanded}
\sum\limits_{k=1}^N \left( \secondderivative{c_k}{t} + \gamma\firstderivative{c_k}{t} + \alpha u \right) \innerproduct{\phi_k}{\phi_j}
=
- \beta^2\sum\limits_{k=1}^N c_k(t) \innerproductdot{\phi_k}{\phi_j}
+
\innerproduct{F}{\phi_j}
\end{equation}

We now let $M$ be the matrix of pairwise inner products of the $\phi_k$'s, for example
\begin{equation} \label{eq:m_definition}
m_{ij} = \innerproduct{\phi_i}{\phi_j}
\end{equation}

And we let $A$ be the matrix of pairwise inner products of $\nabla \phi_k$, for example
\begin{equation} \label{eq:a_definition}
a_{ij} = \innerproductdot{\phi_i}{\phi_j}
\end{equation}

Also, let $\vec{c}(t)$ be the vector with each entry $c_k(t)$ for $k=1, 2, \ldots$. Lastly, define
$\vec{F}(t)$ to be a vector, s.t. the $k$th entry is given as 
\begin{equation} \label{eq:vec_F_definition}
\innerproduct{F}{\phi_k}
\end{equation}

We can now rewrite (\ref{eq:u_h_weak_form_expanded}) as
\begin{equation} \label{eq:u_h_matrix}
M\left( \secondderivative{\vec{c}}{t} + \gamma\firstderivative{\vec{c}}{t} + \alpha \vec{c} \right) = - \beta^2 A \vec{c} + \vec{F}
\end{equation}

We now can build a finite difference scheme for $\vec{c}$ by using the following expansion for the second derivative:
\begin{equation} \label{eq:second_derivative_finite_difference}
\secondderivative{c}{t} \approx \secondfinitediff{c}
\end{equation}
and the following (second order) expansion for the first derivative
\begin{equation} \label{eq:first_derivative_finite_difference}
\firstderivative{c}{t} \approx \firstfinitediff{c}
\end{equation}
these are of course well known, but they can be found in most numerical methods textbooks \cite{difference_formulas}.
It's worth noting that the centered difference was chosen for the first derivative to keep the numerical solution second
order in time.

Substituting into (\ref{eq:u_h_matrix}) gives us
\begin{equation} \label{eq:finite_diff_weak_form}
M \left( \secondfinitediff{c} + \gamma \firstfinitediff{c} + \alpha c^n \right) = - \beta^2 A c^n + \vec{F}
\end{equation}

Lastly, we can rearrange and solve for $c^{n+1}$
\begin{equation} \label{eq:finite_diff_solution}
c^{n+1} = \step{+}^{-1} M^{-1}
\left(
\left( \left(\frac{2}{\delta t^2} - \alpha\right) M - \beta^2 A \right) c^n
-
\step{-} M c^{n-1}
+
\vec{F}
\right)
\end{equation}

We can simplify this expression, mainly in the interest of making the computation faster and more stable. Note
that $\step{+}^{-1} = \frac{2\delta t^2}{2 + \gamma\delta t}$. Expanding out the terms and simplifying gives us
\begin{equation}
c^{n+1}
=
\frac{1}{2 + \gamma\delta t} \left(
(4 - 2\alpha\delta t^2)c^n  - 2\beta^2\delta t^2M^{-1}Ac^n
- (2 - \gamma\delta t)c^{n-1}
+ 2\delta t^2 F
\right)
\end{equation}

Given this formulation, the only remaining challege is the numerical calculation of $M$, $A$ and potentially
$M^{-1}$ (unless it is implicitly calculated every step via a banded matrix solver).

\subsection{Defining $\phi_k$}
Since $\phi_k \in H^1$, we know that it is piecewise linear. We will choose $\phi_k$ to be zero at all nodes
$N_1, N_2, \ldots$ but be 1 at $N_k$. This will cause $\phi_k$ to be tent-shaped. Because $\phi_k$ is piece-wise linear,
in order to find a closed form we need only consider each "piece", which is this case a triangular region.

Consider a single triangle with nodes at $N_k$, $N_j$, and $N_i$. Given our definition of $\phi_k$, we know $\phi_k(N_k) = 1$,
but 0 for any other node. So in the region defined by our triangle (we'll call it $T_{kji}$), $\phi_k$ is non-zero. Because
$\phi_k$ is piece-wise linear, that means $\phi_k$ is planar on the region, and that the plane passes through these three points:
$(x_k, y_k, 1)$, $(x_j, y_j, 0)$, $(x_i, y_i, 0)$. Considering a plane in three dimensions can be written as
\begin{equation} \label{eq:plane}
z = P(x, y) = ax + bx + c
\end{equation}
we can form a linear system to solve for our cooefficients:
\begin{equation} \label{eq:plane_solution}
\mat{
    x_k & y_k & 1 \\
    x_j & y_j & 1 \\
    x_i & y_i & 1
}
\mat{a \\ b \\ c}
=
\mat{1 \\ 0 \\ 0}
\end{equation}
Given the size of the matrix a closed form solution can be used (CITE), or direct solver can be used.

\subsection{Formation of $A$}
The formula for each element of $A$ is given by (\ref{eq:a_definition}). This formulation requires the calculation
of both $\nabla \phi_i$ and $\nabla \phi_j$. However, since every $\phi$ is piece-wise linear, the gradient
is constant (and in most places zero). Within a particular triangle $T_{kji}$, applying the results from
(\ref{eq:plane_solution}) gives
\begin{equation} \label{eq:grad_phi_definition}
\nabla \phi_k = \mat{a_{kji} \\ b_{kji}}
\end{equation}

For any $\phi_k$ and $\phi_j$, $j \ne k$, they will share either zero, one, or two triangular regions. If they share
zero, the gradient of either $\phi_k$ or $\phi_j$ will be zero over the domain, forcing (\ref{eq:a_definition}) to
be zero. If they share only a single region, it is because they lie on the domain boundary, which would imply that
they are zero by the boundary conditions. The last case, sharing two triangular regions implies that they are interior
nodes that are adjacent on the mesh. Because they are adjacent, in the two shared triangular regions both of their
gradients will be non-zero. Furthermore, if $\nabla \phi_k = \mat{a_k \\ b_k}$ and $\nabla \phi_j = \mat{a_j \\ b_j}$
inside of some triangle $T$, we have
\begin{equation}
\int\limits_T \nabla \phi_k \nabla \phi_j
=
\left( a_k a_j + b_k b_j \right) \mathop{Area}(T)
\end{equation}
Summing over both triangles gives a very straightforward method for calculating each individual element of A. Furthermore,
A will be symmetric, since
\begin{equation}
\innerproductdot{\phi_k}{\phi_j} = \innerproductdot{\phi_j}{\phi_k}
\end{equation}
so further computation can be saved.

\subsection{Formation of $M$}
To calculate the elements of $M$, the inner product of corresponding basis functions of $\Omega$ must be calculated
(see (\ref{eq:m_definition})). Since the basis functions are 0 for most of the domain, and most nodes are not
connected, we expect $M$ to be very sparse. So given a triangular mesh, we need only consider two cases: The inner
product of two adjacent interior node basis function (since exterior nodes are 0), or the inner product of an interior
node basis function with itself.

As mentioned before, two adjacent interior node basis functions $\phi_k$ and $\phi_h$ will always share two triangular
regions of $\Omega$ in which they are both non-zero. Considering just one of these triangles $T$, we can find a solution method for a
single region and simply extrapolate up to using two regions. For just the single triangle $T$ defined by the points
$N_k$, $N_j$, and $N_i$, we are wanting to calculate
\begin{equation}
\iint\limits_T \phi_k \phi_j dT
\end{equation}
The problem with attempting to calculate this integral directly is that $T$ may be an very domain to integrate over
directly. Instead, we can apply an affine transformation to $T$, so to transform the region into a $(u,v)$ space that
has more straight-forward limits of integration. We'll refer to the transformation as $S$.

We'll choose $S$ to be a transformation that takes a triangle with corners at $(0,0)$, $(1,0)$, and $(1,1)$. and
transforms them to $T$. We can choose any ordering, but we'll choose $S^{-1}N_k = (0, 0)$, $S^{-1}N_j = (1, 0)$,
and $S^{-1}N_i = (1, 1)$. Note that
by transforming to this new $(u,v)$ space our integral becomes
\begin{equation}
\int\limits_0^1 \int\limits_0^u \phi_k(S(u, v)) \phi_j(S(u, v)) |S| dv du
\end{equation}
using a straight-forward change of variables (CITE).

Since $S$ is affine, we'll need to use homogeneous coordinates to treat as a linear transformation (CITE). Given a homogeneous
formulation, $S$ takes the form
\begin{equation}
\mat{ a & b & c \\ d & e & f \\ 0 & 0 & 1 }
\end{equation}

Given the mapping we chose earlier, we can form an equation
\begin{equation}
S \mat{ 0 & 1 & 1 \\ 0 & 0 & 1 \\ 1 & 1 & 1 }
= \mat{ x_k & x_j & x_i \\ y_k & y_j & y_i \\ 1 & 1 & 1}
\end{equation}
and then rearrange to create a linear system for which we can solve the coefficients of $S$:
\begin{equation}
\mat{
    0 & 0 & 1 & 0 & 0 & 0 \\
    1 & 0 & 1 & 0 & 0 & 0 \\
    1 & 1 & 1 & 0 & 1 & 0 \\
    0 & 0 & 0 & 0 & 0 & 1 \\
    0 & 0 & 0 & 1 & 0 & 1 \\
    0 & 0 & 0 & 1 & 1 & 1 
}
\mat{a \\ b \\ c \\ d \\ e \\ f}
=
\mat{x_k \\ x_j \\ x_i \\ y_k \\ y_j \\ y_i }
\end{equation}
Given the form of the matrix, this is easily solved by hand, and we have
\begin{equation}
S = \mat{
    x_j - x_k & x_i - x_j & x_k \\
    y_j - y_k & y_i - y_j & y_k \\
    0 & 0 & 1
}
\end{equation}
Moreover, $S$ is a simple enough form to find a closed solution for $|S|$:
\begin{equation}
|S| = 
(x_j - x_k)(y_i - y_j) -
(y_j - y_k)(x_i - x_j)
\end{equation}

Lastly, we need to find a closed form of $\phi_k(S(u,v))\phi_j(S(u,v))$, and ultimately a closed form for the full
integral. Using $S$, $\phi_k$ will now take the form
\begin{equation}
\phi_k(S(u,v)) = A_k ((x_j - x_k)u + (x_i - x_j)v + x_k) + B_k((y_j - y_k)u + (y_i - y_j)v + y_k) + C_k
\end{equation}
After some algebra we get
\begin{equation}
((A_k(x_j-x_k) + B_k(y_j-y_k))u + (A_k(x_i-x_j) + B_k(y_i-y_j))v + (A_k x_k + B_k y_k + C_k)
\end{equation}
Or for simplicity we'll right it as
\begin{equation}
A'_k u + B'_k v + C'_k
\end{equation}
Multiplying $\phi_k$ and $\phi_j$ gives us
\begin{equation}
(A'_k A'_j)u^2 + (B_k'B_j')v^2 + (A'_k B'_j + B'_k A'_j)uv + (A'_k C'_j + C'_k A'_j)u
+ (B'_kC'_j + C'_kB'_j)v + (C'_k C'_j)
\end{equation}
Now, we can integrate
\begin{equation}
\int\limits_0^1\int\limits_0^u \phi_k(S(u,v)) \phi_j(S(u,v)) |S| dvdu
\end{equation}
Which gives us a final closed form
\begin{equation}
\left(\frac{A'_k A'_j}{4} + \frac{B_k'B_j'}{12} + \frac{A'_k B'_j + B'_k A'_j}{12} +
\frac{A'_k C'_j + C'_k A'_j + B'_kC'_j + C'_kB'_j}{6} + \frac{C'_k C'_j}{2}
\right) |S|
\end{equation}
for the integral over a single triangle. We can apply this method twice for the two shared triangles to calculate
$\innerproduct{\phi_k}{\phi_j}$.

For $\innerproduct{\phi_k}{\phi_k}$, the method is exactly the same, but we must integrate over all triangles with a
vertex at $N_k$, since $\phi_k$ shares all neighboring triangles with itself.

\subsection{Formation of $M^{-1}$}
\subsection{Formation of $\vec{F}$}
In this simulation $F(x,y,t)$ will be provided by the user in real-time, via mouse clicks. This being said, we have a
choice in how we want to translate a mouse click into a function $F(x,y,t)$. In fact, for simplicity we can bypass
$F(x,y,t)$ altogether and simply consider $\vec{F}$.

In fact, given a mouse click at $(x,y)$ at time $t$, we'll choose the nearest $n$ nodes to $(x,y)$ and subdivide some 
weight $W$ to those nodes proportional to their distance from $(x,y)$. This would mean $n$ of the entries of $\vec{F}$
are non-zero but the rest are zero. In fact, a parameterized form of this weighting can be written as follows
\begin{equation}
w(N_k) = \frac{W}{T|N_k - C| + 1}
\end{equation}
where $N_k$ is some node point, $W > 0$ is some weighting factor, $T > 0$ is a ``tightening term'' and $C$ is the center
point of force application. Note that 1 is added to prevent the denominator from being 0.

\subsection{Error Analysis}
It is certainly useful to know the stability of the scheme chosen in \label{eq:finite_diff_solution}, so that we can
choose appropriate time step values to prevent numerical instability. We'll first consider apply some algebra to
\label{eq:finite_diff_solution}, which gives us a somewhat simpler form
\begin{equation}
c^{n+1} = \left(
    \frac{4}{2 + \gamma\delta t} - \frac{2\beta^2\delta t^2}{2 + \gamma\delta t}
\right) c^n
-
\frac{2 - \gamma\delta t}{2 + \gamma\delta t}c^{n-1}
+
\frac{2\delta t^2}{2 + \gamma\delta t}\vec{F}
\end{equation}

For the purposes of our error analysis, we'll assume that no force is being applied, since in our simulation it's
assumed the user won't provide force the entire time. We'll let $r = \frac{1}{2 + \gamma\delta t}$. Considering the
error $\epsilon^{n+1}$, we have
\begin{equation}
|\epsilon^{n+1}| = |4rI - 2\beta^2\delta t^2rM^{-1}A||\epsilon^n| + (2 - \gamma\delta t)r|\epsilon^{n-1}|
\end{equation}
Worst case, the error did not decrease, meaning $|\epsilon^{n}| = |\epsilon^{n-1}|$, which allows to to consider the
scaling of the scheme
\begin{equation}
\frac{|\epsilon^{n+1}|}{|\epsilon^{n}|}
\le
\frac{4 - 2\beta^2\delta t^2|M^{-1}||A| - 2 + \gamma\delta t}{2 + \gamma\delta t}
\le
1
\end{equation}

After some cleanup, we have
\begin{equation}
\frac{-2\beta^2\delta t^2|M^{-1}||A|}{2 + \gamma\delta t} \le 0
\end{equation}

Since all of the terms are positive, then our scheme is unconditionally stable


\section{Technology}

As much as the mathematics behind this project is interesting, so are the technologies involved. And, as much as the
labor of mathematicians before myself should be cited, the same can be said for developers who have innovated
technologies used in this paper. Most of the resources I've used are open-source, and freely available to use and
modify.

One of the main driving technologies for this is project is WebGL \cite{webgl}, which is an OpenGL ES 2.0 specification
for web browsers. A fairly new technology, the support is quickly growing and is supported in the latest version of
all major desktop browsers, and even some mobile browsers \cite{caniuse_webgl}.

As powerful as the OpenGL standard is, a library certainly can eliminate a good chunk of boilerplate. Three.js
\cite{threejs} is \emph{by far} the most powerful library for this, and provides a wide array of tools for 3D
graphics work, including cameras, meshes, materials, and basic linear algebra operations like cross and dot products.

DAT-GUI \cite{datgui} was used for providing real-time parameter adjustment.

From math side of things, numeric.js \cite{numericjs} was invaluable for providing most of the linear algebra routines
used, such as LU factorization, and matrix-vector products. Unfortunately, the library is not actively developed, but is
still very useful, in that it provides most of the useful linear algebra routines needed by most applications, such as
SVD, and even sparse matrix representations as well.

For triangulation the Delaunay triangulation algorithm \cite{triangulation} was used. An implementation by Mikola
Lysenko existed already \cite{delaunay}, so I used that one. To build the mesh, points were sampled in the domain,
and filtered by polygon containment. The reduced point set was then triangulated via the algorithm.


\bibliographystyle{plain}
\bibliography{main}

\end{document}
