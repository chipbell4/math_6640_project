\documentclass[a4paper,12pt]{article}

\usepackage{amsmath, amsthm, amssymb, url, hyperref}

\begin{document}

% some custom commands
\newcommand{\innerproduct}[2]{\int\limits_{\Omega} #1 #2 d\Omega}
\newcommand{\innerproductdot}[2]{\int\limits_{\Omega} \nabla #1 \cdot \nabla #2 d\Omega}
\newcommand{\firstderivative}[2]{\frac{d #1}{d #2}}
\newcommand{\secondderivative}[2]{\frac{d^2 #1}{d #2^2}}
\newcommand{\firstpartial}[2]{\frac{\partial #1}{\partial #2}}
\newcommand{\secondpartial}[2]{\frac{\partial^2 #1}{\partial #2^2}}
\newcommand{\laplacian}[1]{\Delta #1}
\newcommand{\secondfinitediff}[1]{\frac{#1^{n+1} - 2#1^n + #1^{n-1}}{\delta t^2}}
\newcommand{\firstfinitediff}[1]{\frac{#1^{n+1} - #1^{n-1}}{2\delta t}}
\newcommand{\step}[1]{\left( \frac{1}{\delta t^2} #1 \frac{\gamma}{2\delta t} \right)}
\newcommand{\mat}[2][rrrrrr]{
    \left(\begin{array}{#1}
    #2 \\
    \end{array}
    \right)
}

\title{A Real Time Finite Element Simulation of a Damped/Forced Surface Wave on a Membrane using WebGL}
\author{Chip Bell}
\date{December 2014}
\maketitle

\begin{abstract}
In this project, a finite element model is developed for a modified wave equation that includes both a damping term
and a forcing term. A demonstration was built using various modern web technologies, including JavaScript and WebGL.
Later sections describe those technologies, explain how they were used in building the simulation, and mention some
of the issues encountered, along with how they were overcome or avoided altogether.
\end{abstract}


\section{The main PDE}
For my project I'll be modeling a modified form of the wave equation that encorporates damping, elasticity, and
forcing. I'll solve the equation over a domain $\Omega$ in 2 dimensions with a Dirichlet boundary $\Gamma$. 
\begin{equation} \label{eq:main_pde}
\secondpartial{u}{t} + \gamma \firstpartial{u}{t} + \alpha u
=
\beta^2 \laplacian{u} + F(x,y,t)
\end{equation}
where $u$ is the wave height at some point $(x,y)$ at some time $t$, $\gamma \ge 0$ is a damping factor for the wave,
$\beta$ is the wave speed, $\alpha$ is an elasticity term and $F$ is a forcing function on the wave. This equation
will model a taunt drum head over the domain $\Omega$, our intended purpose but is also known as the
\emph{Telegraph Equation} and has analytic solutions \cite{pde_solution}. Most derivations I have found approach
the problem from a electric circuit standpoint. One such derivation for the one dimensional case can be found in
\cite{pde_derivation}.

\section{Finite Element Analysis}
\input{fem/derivation}
\subsection{Defining $\phi_k$}
Since $\phi_k \in H^1$, we know that it is piecewise linear. We will choose $\phi_k$ to be zero at all nodes
$N_1, N_2, \ldots$ but be 1 at $N_k$. This will cause $\phi_k$ to be tent-shaped. Because $\phi_k$ is piece-wise linear,
in order to find a closed form we need only consider each "piece", which is this case a triangular region.

Consider a single triangle with nodes at $N_k$, $N_j$, and $N_i$. Given our definition of $\phi_k$, we know $\phi_k(N_k) = 1$,
but 0 for any other node. So in the region defined by our triangle (we'll call it $T_{kji}$), $\phi_k$ is non-zero. Because
$\phi_k$ is piece-wise linear, that means $\phi_k$ is planar on the region, and that the plane passes through these three points:
$(x_k, y_k, 1)$, $(x_j, y_j, 0)$, $(x_i, y_i, 0)$. Considering a plane in three dimensions can be written as
\begin{equation} \label{eq:plane}
z = P(x, y) = ax + bx + c
\end{equation}
we can form a linear system to solve for our cooefficients:
\begin{equation} \label{eq:plane_solution}
\mat{
    x_k & y_k & 1 \\
    x_j & y_j & 1 \\
    x_i & y_i & 1
}
\mat{a \\ b \\ c}
=
\mat{1 \\ 0 \\ 0}
\end{equation}
Given the size of the matrix a closed form solution can be used (CITE), or direct solver can be used.

\subsection{Formation of $A$}
The formula for each element of $A$ is given by (\ref{eq:a_definition}). This formulation requires the calculation
of both $\nabla \phi_i$ and $\nabla \phi_j$. However, since every $\phi$ is piece-wise linear, the gradient
is constant over a single triangular region (and in most places zero). Within a particular triangle $T_{kji}$, applying the results from
(\ref{eq:plane_solution}) gives
\begin{equation} \label{eq:grad_phi_definition}
\nabla \phi_k = \mat{a_{kji} \\ b_{kji}}
\end{equation}

For any $\phi_k$ and $\phi_j$, $j \ne k$, they will share either zero, one, or two triangular regions. If they share
zero, the gradient of either $\phi_k$ or $\phi_j$ will be zero over the domain, forcing (\ref{eq:a_definition}) to
be zero. If they share only a single region, it is because they lie on the domain boundary, which would imply that
they are zero by the boundary conditions. The last case, sharing two triangular regions implies that they are interior
nodes that are adjacent on the mesh. Because they are adjacent, in the two shared triangular regions both of their
gradients will be non-zero. Furthermore, if $\nabla \phi_k = \mat{a_k \\ b_k}$ and $\nabla \phi_j = \mat{a_j \\ b_j}$
inside of some triangle $T$, we have
\begin{equation}
\int\limits_T \nabla \phi_j \cdot \nabla \phi_k dT
=
\left( a_k a_j + b_k b_j \right) \mathop{Area}(T)
\end{equation}
Summing over both triangles gives a very straightforward method for calculating each individual element of A. Furthermore,
A will be symmetric, since
\begin{equation}
\innerproductdot{\phi_k}{\phi_j} = \innerproductdot{\phi_j}{\phi_k}
\end{equation}
so further computation can be saved.


\subsection{Formation of $M$}
\subsection{Formation of $M^{-1}$}
\subsection{Formation of $\vec{F}$}

\section{Technology}

As much as the mathematics behind this project is interesting, so are the technologies involved. And, as much as the
labor of mathematicians before myself should be cited, the same can be said for developers who have innovated
technologies used in this paper. Most of the resources I've used are open-source, and freely available to use and
modify.


\bibliographystyle{plain}
\bibliography{main}

\end{document}
