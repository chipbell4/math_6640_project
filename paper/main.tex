\documentclass[a4paper,12pt]{article}

\usepackage{amsmath, amsthm, amssymb}

\begin{document}

\section{The main PDE}
The partial differential equation models a damped membrane over a region $\Omega$ bounded by a closed curve $\Gamma$.
\begin{equation} \label{eq:main_pde}
\frac{\partial^2u}{\partial t^2} + \gamma \frac{\partial u}{\partial t}
=
\beta^2 \Delta u + F(x,y,t)
\end{equation}

where $u$ is the wave height at some point $(x,y)$ at some time $t$, $\gamma$ is a damping factor for the wave,
$\beta$ is the wave speed, and $F$ is a forcing function on the wave. Because this a drum head, (or ``membrane''), we choose
$u$ along $\Gamma$ to be 0, giving us a Dirichet boundary condition.

\section{Derivation of the Finite Element Solution}
We first consider the weak form of (\ref{eq:main_pde}) by approximating $u$ as $u_h \in H^1(\Omega)$. Assuming we have
a triangulation of $\Omega$, consisting of $N$ triangles, we can choose 
$\phi_1(x,y), \phi_2(x,y), \ldots, \phi_N(x,y)$ as basis functions for $H^1(\Omega)$, we can write $u_h$ as linear
combination of those basis elements:
\begin{equation} \label{eq:u_h}
u_h(x,y,t) = \sum\limits_{k=1}^N c_k(t) \phi_k(x,y)
\end{equation}
Note that at a particular time $t$, $c_k(t)$ will give the height of a wave at node $k$. So, by solving each of the values
of $c_k(t)$ over time, we will obtain the values of $u_h$ at time $t$.

The weak form of $u_h$ is given by
\begin{equation} \label{eq:u_h_weak_form}
\int\limits_{\Omega} 
    \left( 
        \frac{\partial^2u_h}{\partial t^2} + \gamma \frac{\partial u_h}{\partial t}
    \right)
    v(x,y) d\Omega
    =
    \beta^2 \int\limits_{\Omega} \left( \Delta u \right) v(x,y) d\Omega
    + \int\limits_{\Omega} F(x,y,t)v(x,y) d\Omega
\end{equation}
for any $v \in H^1(\Omega)$. Since this works for any $v$, we can choose $v$ to be any of the basis functions $\phi$
we chose for $H^1(\Omega)$. So, for any basis $\phi_j$, we can expand $u_h$ using (\ref{eq:u_h}) to obtain a new form:
\begin{equation} \label{eq:u_h_weak_form_expanded}
\sum\limits_{k=1}^N \left( \frac{d^2c_k}{dt^2} + \gamma\frac{dc_k}{dt} \right) \int\limits_{\Omega} \phi_k\phi_j d\Omega
=
\beta^2\sum\limits_{k=1}^N c_k(t) \int\limits_{\Omega} \nabla \phi_k \cdot \nabla \phi_j d\Omega
+
\int\limits_{\Omega} F\phi_j d\Omega
\end{equation}

We now let $M$ be the matrix of pairwise inner products of the $\phi_k$'s, for example
\begin{equation} \label{eq:m_definition}
m_{ij} = \int\limits_{\Omega} \phi_i \phi_j d\Omega
\end{equation}

And we let $A$ be the matrix of pairwise inner products of $\nabla \phi_k$, for example
\begin{equation} \label{eq:a_definition}
a_{ij} = \int\limits_{\Omega} \nabla \phi_i \cdot \nabla \phi_j d\Omega
\end{equation}

Also, let $\vec{c}(t)$ be the vector with each entry $c_k(t)$ for $k=1, 2, \ldots$. Lastly, define
$\vec{F}(t)$ to be a vector, s.t. the $k$th entry is given as 
\begin{equation} \label{eq:vec_F_definition}
\int\limits_{\Omega} F(x,y,t) \phi_k(x,y) d\Omega
\end{equation}

We can now rewrite (\ref{eq:u_h_weak_form_expanded}) as
\begin{equation} \label{eq:u_h_matrix}
M\left( \frac{d^2\vec{c}}{dt^2} + \gamma\frac{d\vec{c}}{dt} \right) = \beta^2 A \vec{c} + \vec{F}
\end{equation}

We now can build a finite difference scheme for $\vec{c}$ by using the following expansion for the second derivative:
\begin{equation} \label{eq:second_derivative_finite_difference}
\end{equation}

\end{document}
