\documentclass[a4paper,12pt]{article}

\usepackage{amsmath, amsthm, amssymb}

\begin{document}

\section{The main PDE}
The partial differential equation models a damped membrane over a region $\Omega$ bounded by a closed curve $\Gamma$.
\begin{equation} \label{eq:main_pde}
\frac{\partial^2u}{\partial t^2} + \gamma \frac{\partial u}{\partial t}
=
\beta^2 \nabla u + F(x,y,t)
\end{equation}

where $u$ is the wave height at some point $(x,y)$ at some time $t$, $\gamma$ is a damping factor for the wave,
$\beta$ is the wave speed, and $F$ is a forcing function on the wave. Because this a drum head, (or ``membrane''), we choose
$u$ along $\Gamma$ to be 0, giving us a Dirichet boundary condition.

\section{Derivation of the Finite Element Solution}
We first consider the weak form of (\ref{eq:main_pde}) by approximating $u$ as $u_h \in H^1(\Omega)$. Assuming we have
a triangulation of $\Omega$, consisting of $N$ triangles, we can choose 
$\phi_1(x,y), \phi_2(x,y), \ldots, \phi_N(x,y)$ as basis functions for $H^1(\Omega)$, we can write $u_h$ as linear
combination of those basis elements:
\begin{equation} \label{eq:u_h}
u_h(x,y,t) = \sum\limits_{k=1}^N c_k(t) \phi_k(x,y)
\end{equation}
Note that at a particular time $t$, $c_k(t)$ will give the height of a wave at node $k$. So, by solving each of the values
of $c_k(t)$ over time, we will obtain the values of $u_h$ at time $t$.

\end{document}
