\section{Instructions}
The project is hosted on my personal website \cite{hosted_site}. The source control for the project is hosted on Github
\cite{github}. Follows is a basic set of instructions and description of how the application works.

The first screen allows the user to draw a polygon (defaulting to a square). When a user presses the key the polygon is
triangulated and the simulation begins. Triangulation is accomplished by first converting the polygon to a convex form,
then sampling latice points for point containment within the polygon. Once this point set has been obtained, is
triangulated using the Delaunay triangulation and converted into a mesh.

Using the mesh, the mass and stiffness matrices are calculated. The LU factorization of the mass matrix is also
calculated and stored for later operations requiring the inverse. Once these matrices are calculated, the numerical
scheme is stepped in real-time.

To add energy to the system, clicking on the mesh applies force to the system. This force is calculated by an inverse
square relationship to the click center, and adjusting the click weight and tightness will adjust the amount of force
and spread of force respectively. Also, other parameters of the system are adjustable, including wave speed,
elasticity, and damping.

\subsection{Current Standing Issues}
No software is perfect, and the same can be said for this application. Currently, some polygons after being triangulated
simply blow up when force is applied. I'm almost positive that this is caused by triangles that are too small, which
causes the mass matrix entries to be small. This, in turn, causes any LU solving with the mass matrix to have very large
numbers, which essentially ``ghosts'' a very large first derivative into the system due to numerical round-off. The wave
then shoots off to infinity.

I'm currently still resolving this issue as of the time of writing. My solution will essentially be to remove interior
points that would cause small triangles, i.e. points that are very close (within some tolerance) of the border of
polygon. Currently, the best solution is to simply refresh the page.
