\section{Technology}

As much as the mathematics behind this project is interesting, so are the technologies involved. And, as much as the
labor of mathematicians before myself should be cited, the same can be said for developers who have innovated
technologies used in this paper. Most of the resources I've used are open-source, and freely available to use and
modify.

One of the main driving technologies for this is project is WebGL \cite{webgl}, which is an OpenGL ES 2.0 specification
for web browsers. A fairly new technology, the support is quickly growing and is supported in the latest version of
all major desktop browsers, and even some mobile browsers \cite{caniuse_webgl}.

As powerful as the OpenGL standard is, a library certainly can eliminate a good chunk of boilerplate. Three.js
\cite{threejs} is \emph{by far} the most powerful library for this, and provides a wide array of tools for 3D
graphics work, including cameras, meshes, materials, and basic linear algebra operations like cross and dot products.
