\section{Technology}

As much as the mathematics behind this project is interesting, so are the technologies involved. And, as much as the
labor of mathematicians before myself should be cited, the same can be said for developers who have innovated
technologies used in this paper. Most of the resources I've used are open-source, and freely available to use and
modify.

One of the main driving technologies for this is project is WebGL \cite{webgl}, which is an OpenGL ES 2.0 specification
for web browsers. A fairly new technology, the support is quickly growing and is supported in the latest version of
all major desktop browsers, and even some mobile browsers \cite{caniuse_webgl}.

As powerful as the OpenGL standard is, a library certainly can eliminate a good chunk of boilerplate. Three.js
\cite{threejs} is \emph{by far} the most powerful library for this, and provides a wide array of tools for 3D
graphics work, including cameras, meshes, materials, and basic linear algebra operations like cross and dot products.

DAT-GUI \cite{datgui} was used for providing real-time parameter adjustment.

From math side of things, numeric.js \cite{numericjs} was invaluable for providing most of the linear algebra routines
used, such as LU factorization, and matrix-vector products. Unfortunately, the library is not actively developed, but is
still very useful, in that it provides most of the useful linear algebra routines needed by most applications, such as
SVD, and even sparse matrix representations as well.

For triangulation the Delaunay triangulation algorithm \cite{triangulation} was used. An implementation by Mikola
Lysenko existed already \cite{delaunay}, so I used that one. To build the mesh, points were sampled in the domain,
and filtered by polygon containment. The reduced point set was then triangulated via the algorithm.

In an effort to keep code quality in check, a mocha \cite{mocha} was used as the testing framework, with chai
\cite{chai} as the assertion library. Code linting was performed via jshint \cite{jshint}. Package bundling was
performed via browserify \cite{browserify} to keep code organized. Lastly, gulp \cite{gulp} was used a task runner
and build tool.

Lastly, all of the packages are developmented and maintained via Node.js \cite{node} a JavaScript runtime for the server,
and NPM \cite{npm}, the de facto package manager for Node.
